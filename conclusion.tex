\section{Conclusion and Future Works}
本論文では、POMDP の問題設定において良質な特徴表現を得るために、ニューラルネットワーク上の各ユニットをエージェントとして扱うフレームワーク、NaaA について述べた。
NaaA のフレームワークでは、ジレンマ問題を解決し、それぞれのエージェントの持つ付加価値がナッシュ均衡として得られ、全体としてパレート最適になることを示した。
入札価格の決定アルゴリズムの一つとして、$Q$-learning に基づくネットワーク Valuation Net を示した。
評価実験では、Atari と VizDoom を用いた実験を行い、実験結果が既存手法よりもよくなることを示した。

% 今後の方向性は思いついたら書き足していく
今後の方向性として、高速化、Valuation Net を A3C などの on-policy な手法で置き換えるといった方向性の他、神経科学的な説明を可能にしていくといった方法、遺伝的アルゴリズムとの組み合わせが挙げられる。
