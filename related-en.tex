\section{Related Work}
Training neural network with multi-agent game is emerging methodology.
Generative adversarial nets (GAN) \citep{goodfellow2014generative} has goal to obtain true generative distribution as Nash equilibrium of a competitive game made of two agents with contradicting rewards: a generator and a discriminator. 
In game theory, the outcome maximizing overall reward is named Pareto optimality.
Nash equilibrium is not guaranteed to converge Pareto optimality, and difference of the both is named dilemma.
As existence of dilemma depends on the reward design, the method to resolve the dilemma with good reward design is being researched: mechanism design \citep{myerson1983mechanism} also known as inverse game theory.
Mechanism design is applied to auction \citep{vickrey1961counterspeculation} and matching \citep{gale1962college}.
GAN and our proposal, NaaA, are outcome from mechanism design.
NaaA applies digital goods auction \citep{guruswami2005profit} to reinforcement learning with multi-agent neural network, 
and obtain maximized return by units as Nash equilibrium.

NaaA belongs to a class of partially observable stochastic game (POSG) \citep{hansen2004dynamic} as it processes multiple units as agents.
POSG is a class of reinforcement learning in which multiple agents in a POMDP environment, and it has several research issues.
The one is communication.
CommNet \citep{sukhbaatar2016learning} exploits the characteristics of a unit which agnostic to topology of other units, it employs backpropagation to training multi-agent communication.
The another one is credit assignment.
Instead of reward $R(a_t)$ of an agent $i$ for actions at $t$ $a_t$, 
QUICR-learning \citep{agogino2006quicr} maximizes counterfactual reward $R(a_t) - R(a_t - a_{it})$, the difference in the case of the agent $i$ takes an action $a_{it}$ ($a_t$) and not ($a_t-a_{it}$).
COMA \citep{foerster2017counterfactual} also maximizes counterfactual reward in a setting of actor-critic.
In the setting, all the actors has common critic, and improves the both actors and critic with time difference (TD)-error of counterfactual reward.
This paper unifies both the issues, communication and credit assignment.
The main proposal is framework to manage the agents to maximize {\em counterfactual return}, the extended counterfactual reward along with time axis.

TODO: Dropconnect
