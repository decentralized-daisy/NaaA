\appendix
\section*{Appendix}

% Set numbering of section alphabet: A, B, C,...
\setcounter{section}{1}
\renewcommand{\thesection}{\Alph{section}}

% Contents is started from here
%\subsection{Proof}
%We first proof $\rho_{ijt} = 0$ is the Nash equilibrium.
%The value function can be written as $r_{it} - c_{it} + \gamma V_{i,t+1}$.
%At the point, as other agents plays $\rho_{ijt} = 0$, the value function is $- c_{it}$.
%To maximizing this equation, $\rho_{ijt} = 0$ holds. 

\subsection{Proof of Theorem \ref{thm:optimal-bidding}}
%As for a buyer, the asking price $\ask$ for a seller is unknown,
%we address $\ask$ which has support $[0, \infty)$,
%and consideration to maximize $\Expect{q}{G(b,q)}$,
%In this case, the following equation holds.
%\begin{flalign}
%\deriv{b}{}\Expect{q}{G(b,q)} 
%&= \deriv{b}{}\int_0^\infty (H(b - q) \cdot (v-q) + G_0) p(q) dq \notag \\
%&= \deriv{b}{} \left[ \int_0^b (v-q) p(q) dq + G_0 \int_0^\infty p(q)dq \right] \notag \\
%&= \deriv{b}{} \int_0^b (v-q) p(q) dq \notag \\
%&= (v-b) p(q=b), \notag 
%\end{flalign}
%Therefore, the condition to maximize $\Expect{q}{G(b,q)}$ is $b=v$.

The optimization problem in Eq:\ref{Eq:optimization-probem} is made of two terms except of the constant, 
and the only second term is depends on $\vect{b}$.
Hence, we consider to optimize the second term.
The optimal bidding prices $\hat{\vect{q}}_t$ is given by the following equation.

\begin{flalign}
\hat{\vect{b}}_{it} 
&= \argmin_{\vect{b}} \Expect{\vect{q}_t}{\vect{g}_{it}(\vect{b})^\T( \vect{q}_t - \gamma \vect{o}_{it}  )} 	
= \argmin_{\vect{b}} \Expect{\vect{q}_t}{H(\vect{b} - \vect{q}_{t})^\T ( \vect{q}_t - \gamma \vect{o}_{it}  )}	\notag \\
&= \argmin_{\vect{b}} \Expect{\vect{q}_t}{\sum_{j=1}^N H(b_{j} - q_{jt}) ( q_{jt} - \gamma o_{ijt}  )}		
= \argmin_{\vect{b}} \sum_{j=1}^N \Expect{q_{jt}}{H(b_{j} - q_{jt}) ( q_{jt} - \gamma o_{ijt}  )},
\end{flalign}

From independence, the equation is solved if we solve the following problem.

\begin{flalign}
\hat{b}_{ijt} = \argmin_{b} \Expect{q_{jt}}{H(b - q_{jt}) ( q_{jt} - \gamma o_{ijt}  )}, \, \, \, \forall j \in \{1, \dots, N\}
\end{flalign}

Hence, $\hat{b}_{ijt}$ can be derived as the solution which satisfies the following equation. 
\begin{flalign}
	\left. \deriv{b}{}\Expect{q_{jt}}{H(b - q_{jt}) ( q_{jt} - \gamma o_{ijt}  )} \right|_{b=\hat{b}_{ijt}}  = 0,  \, \, \, 
\left. \derivN{2}{b}{}\Expect{q_{jt}}{H(b - q_{jt}) ( q_{jt} - \gamma o_{ijt}  )}\right|_{b=\hat{b}_{ijt}} > 0 \notag 
\end{flalign}

For simplicity, we let $q = q_{jt}$ and $o = o_{ij,t+1}$. Then, the following equation holds.

\begin{flalign}
\deriv{b}{}\Expect{q}{H(b - q)( q - \gamma o  )} 
&= \deriv{b}{}\int_0^\infty H(b - q)(q - \gamma o) p(q) dq \notag \\
&= \deriv{b}{} \int_0^b (q - \gamma o) p(q) dq \notag \\
	&= (b-\gamma o) p(q=b), \label{eq:deriv1}
\end{flalign}
\begin{flalign}
	\derivN{2}{b}{}\Expect{q}{H(b - q)( q - \gamma o  )} = p(q=b) + (b - \gamma o) \deriv{b}{}p(q=b)  \label{eq:deriv2}
\end{flalign}

Then, condition $\deriv{b}{}\Expect{q_{jt}}{H(b - q_{jt}) ( q_{jt} - \gamma o_{ijt}  )} |_{b=\hat{b}_{ijt}}  = 0$ is satisfied only by $\hat{b}_{ijt} = \gamma o_{ijt}$.
We substitue $\hat{b}_{ijt}$ into Eq:\ref{eq:deriv2},
\begin{flalign}
\left. \derivN{2}{b}{}\Expect{q_{jt}}{H(b - q_{jt}) ( q_{jt} - \gamma o_{ijt}  )}\right|_{b=\hat{b}_{ijt}} 
	&= p(q=\hat{b}_{ijt}) + 0 > 0
\end{flalign}

Therefore, $\hat{b}_{ijt} = \gamma o_{ijt}$ is a unique solution as the minimum point.
From generality, $\hat{\vect{b}}_{it} = \gamma \vect{o}_{it}$ holds.
