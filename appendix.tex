\appendix
\section*{Appendix}
\setcounter{section}{1}
\renewcommand{\thesection}{\Alph{section}}
\subsection{定理\ref{thm:optimal-bidding}の証明}
買い手の獲得する生涯報酬 $G$ は次で与えられる。
\begin{flalign}
G(b,q) = g(b,q) \cdot (v-q) + G_0,
\end{flalign}
ただし、$g$ は割当(allocation)であり、$G_0$ はユニットを購入しなかった場合の生涯報酬である。割当は、$g(b, q) = H(b - q)$ が成立する。$H$ はステップ関数である。

買い手にとって売り手が提示する asking price $\ask$ は未知であるため、
$\ask$ を台 $[0, \infty)$ の上の確率変数であるとして扱い、
期待値 $\Expect{q}{G(b,q)}$ を最大化することを考える。
このとき、
\begin{flalign}
\deriv{b}{}\Expect{q}{G(b,q)} 
&= \deriv{b}{}\int_0^\infty (H(b - q) \cdot (v-q) + G_0) p(q) dq \notag \\
&= \deriv{b}{} \left[ \int_0^b (v-q) p(q) dq + G_0 \int_0^\infty p(q)dq \right] \notag \\
&= \deriv{b}{} \int_0^b (v-q) p(q) dq \notag \\
&= (v-b) p(q=b) \notag 
\end{flalign}
したがって、 $\Expect{q}{G(b,q)}$ が最大となるための条件は $b=v$ ある。
