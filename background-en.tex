\section{Background}
First, we consider a POMDP environment in which a single agent acts.
The POMDP environment is a seven-tuple $(\S, \A, \Trans, \R, \Observations, \ObProb, \gamma)$,
where $\S$ represents a set of states, $\A$ stands for a set of actions, $\Trans$ denotes a transitive probability, 
$\Observations$ represents a possible set of observations, $\ObProb$ denotes a set of observation probability, and
$\gamma$ is the discount rate.
An agent partially predicts state $h \in \S$ through an observation $s \in \Observations$.
Generally, $s$ has higher dimensions than $h$, and is complex.
For example, although Atari 2600 has a read only memory (RAM) as the true state, which contains 128 bytes,
the generated image from that $s$ has more than 10,000 dimensions.
Therefore, DQN and DRQN abstract $s$, and create original state representation to predict good action efficiently.
(Although the original paper of DQN assumes MDP, the paper of DRQN pointed out that the environment is POMDP).
Although DQN does not address the state transition directly because it is model-free method, 
some interpretations hold that the hidden state representation is learned in the previous layer of the output layer \citep{zahavy2016graying}
Using the method below, we assume that the agent chooses an action through a neural network.

The POSG environment is a multi-agent environment defined by a tuple $(\S, \A^i, \Trans, \R^i, \Observations^i, \ObProb^i, \gamma^i)_{i \in \mathcal{I}}$,
where $\mathcal{I}$ is a finite set of agents indexed 1, \dots, $N$, $\S$ represents a set of states, $\A^i$ stands for a set of actions, $\Trans$ denotes a transitive probability, 
$\R^i: \S \times \A^1 \times \cdots \times \A^N \rightarrow \Real $ is a function from the state and the action of all the agent to rel value.
$\Observations^i$ represents a possible set of observations, $\ObProb^i$ denotes a set of observation probability, and
$\gamma^i$ is the discount rate.
Each agent has an policy $\pi_i: \Observations^i \rightarrow \A^i$, and they maximize their return by interacting the environment.
What is differ than POMDP is there are $N$-agents.

We employ several concept from game theory.
Although RL and game theory are typically investigated in parallel, several concepts in game theory can be written in domain of RL.
The (Bayesian) Nash equilibrium $\hat{\pi}_{i}$ is a policy that all the agent maximize their expected reward. That is,
\begin{flalign}
\hat{\pi}(s_{it}) = \argmax_{a_i \in \A^i} \Expect{ \vect{a}_{-i} \in A^{-i}, h \sim \ObProb^i(\cdot) }{ \R^i(h, \vect{a}) | s_{it}} \, \forall i \in \mathcal{I},
\end{flalign}
where $\mathcal{A}^{-i}$ is a set of actions except of $i$. 
Intuitively, the equation took an expected value of reward to integrate out unobserved other agents' action.
As the Nash equilibrium is enough to state only the best action in the most cases, we use the notation with action $\hat{\vect{a}}$ in the following.

The design of NaaA is inspired by neuroscience.
A neuron in a neurocircuit consumes adenosine triphosphate (ATP) supplied from connected astrocytes.
The astrocyte is a glia cell, which forms the structure of a brain. It supplies fuel from the vessel.
Because the amount of ATP is constrained, the discarded neuron will become extinct with execution of apoptosis.
Also, because apoptosis of a neuron is restrained by neurotrophins (NTFs) such as nerve growth factor (NGF) and brain-derived neurotrophic factor (BDNF),
neurons which can obtain much NTF will live.
The perspective of interpreting a neuron as an independent living object is known as neural Darwinism \citep{edelman1987neural}.
