\section{Discussion}
\subsection{Disadvantage}
Disdvantage としてまず挙げられるのは計算量である。
Envy-free auction では需要の計算にソートの演算が入るために、
直列化しなければならない箇所があるため、
これらについては近似を行うなどして改善していく必要がある。

個別の最適化技術について述べると、
Envy-free auction は、買い手のエージェント同士の価格がわからない sealed な状態であれば、
正直性(truthfulness)が成り立つが、一方で買い手同士がコミュニケーションを行い
価格を共有し合う状態においては、買い手が自由に価格を偽装できることが知られている。
これについては、によって解決方法が示されている。

Valuation Net は、用いるニューラルネットワークによっては実装が困難であることがある。
これは著者らの GitHub に Linear と CNN は公開しているが、
RNN などについては今後の研究課題となる。

\subsection{Application}
NaaA は、ネットワークが分散されている環境での学習や、サブモジュールでの制御に有用である。
具体的に、以下の技術に応用が可能である。

\begin{itemize}
\item ハイパーパラメータチューニング。Neuroevolution など、遺伝的アルゴリズムを用いてハイパーパラメータチューニングを用いるアルゴリズムがすでにいくつか提案されている。このとき、fitness 関数として利益を用いることで、より強化学習の目的に特化したニューラルネットワークを得ることができると考えられる。
\item アテンション制御。一部のアテンションの研究では、強化学習を用いてアテンションの制御を行っている。
\item アンサンブル。複数のモデルの混合に今回の技術を用いることができる。
\end{itemize}


\if0
(作成中)以下のトピックについて言及

何を観測するか考えるという意味合いにおいてはアテンションの拡張である

送金方法

通信スピードの問題

実際の POMDP への拡張

実装

オークションの competitive 性

\subsection{NaaA における「付加価値」とは何か}
具体的な付加価値の例は、統合( $\vect{w}^\T \vect{x}$ )、増幅・減衰( $cx$ )、整流( $\mathrm{ReLU}(x)$ )、バイアス ($x+b$) などである。

\subsection{信頼との関係}
\subsection{価値評価・買い物との関係}

\subsection{Disadvantage}
計算量の問題がある。

\subsection{アプリケーション}
NaaA は、分散環境で何かやったり、サブモジュールで何かコントロールするような場合に有用である。

・ハイパーパラメータチューニング(遺伝的アルゴリズムを用いてハイパーパラメータをチューニングする)
	・Neuroevolution。報酬系が fitness を決める。
・自己組織化ニューラルネットワーク
・アテンション制御
・アンサンブル法
\fi


\if0
想定される反論
・本質コストや NOOP の存在は、パレート劣位の証左にならない: オークションを使わなくても、最低コストだけ払えばよいのでは?
・
\fi

