\section{Neuron as an Agent}
%TODO: Show the figure.

A typical artificial neural network is a directed graph $\neuralnet = (\units, \edges)$ among the units.
$\units = \{\unitAt{1}, \dots, \unitAt{N}\}$ is a set of the units. $\edges \subset \units^2$ is a set of edges representing connections between two units.
If $(\unit, \unitAt{j}) \in \edges$, then connection $\unit \rightarrow \unitAt{j}$ holds, indicating that $\unitAt{j}$ observes activation of $\unit$.
We denote activation of the unit $\unit$ at time $t$ as $x_{it} \in \Real$.
Additionally, we designate a set of units which unit $i$ connects to as $\followers = \{j | (\unit, \unitAt{j}) \in \edges \}$ and a set of units which unit $i$ is connected from as $\followees = \{j | (\unitAt{j}, \unit) \in \edges \}$.
We denote $\friends = \followees \cup \followers$.

NaaA interprets $\unit$ as an agent.
Therefore, $\neuralnet$ is a multi-agent system.
An environment for $\unit$ comprises an environment that the multi-agent system itself touches and a 
set of the unit to which $\unit$ directly connects: $\{v_i \in V | i \in \friends\}$.
We distinguish both environments by naming the former as an external environment, and by naming the latter as an internal environment.
$\unit$ will receive rewards from both environments.
We add the following assumption for characteristics of the $\unit$.
\begin{enumerate}
\renewcommand{\labelenumi}{N\arabic{enumi}:}
\item (Selfishness) 
	Instead of minimizing the global training error,
	at each timing $t$, $\unit$ acts to maximize toward maximizing its own return (cumulative discounted reward)
	$G_{it} = \sum_{k=0}^T \gamma^k \rewardAt{i,t+k}$, where $\gamma \in [0, 1]$ is the discount rate and $T$ is the terminal time.
\item (Conversation) 
	The summation of a reward by which $\units$ will receive both an internal and external environment $\reward$ over 
	all the units are equivalent to reward $R_t^{\mathrm{ex}}$, which the entire multi-agent system receives from 
	the external environment.
\item (Trade) 
	The $\unit$ receives internal reward $\rho_{jit}$ from $\unitAt{j} \in \units$ in exchange of activation signal $x_i$
	before transferring the signal to the unit. At the same time, $\rho_{jit}$ is subtracted from the reward of $v_j$.
\item (NOOP) 
	$\unit$ has NOOP (no operation), for which the return is $\delta > 0$ as an action.
	With NOOP, the unit inputs nothing and outputs nothing.
\end{enumerate}
In terms of neuroscience,
N1 states that the unit acts as a cell.
N2 and N3 state the distribution of NTF. N4 corresponds to apoptosis.
NOOP is selected when the expected returns of the other actions are non-positive.
In the following, we construct the framework of NaaA from the assumptions.

\subsection{Cumulative Discounted Profit Maximization Framework}
We denote the external reward by which unit $\unit$ receives at time step $t$ as $R_{it}^\mathrm{ex}$, where $\sum_{i=1}^n R_{it}^\mathrm{ex} = R_t^{\mathrm{ex}}$ holds.
From N3, reward $R_{it}$, which $\unit$ receives at $t$ can be written as the following.
\begin{flalign}
	R_{it} = 
	  R_{it}^\mathrm{ex} + \sum_{j \in N^\mathrm{out}_i} \rho_{jit} 
	- \sum_{j \in N^\mathrm{in}_i} \rho_{ijt}.
\end{flalign}
The equation is divided into positive terms and a negative term, we name the former as revenue, and the latter as cost, and denote them respectively as $r_{it} = R_{it}^\mathrm{ex} + \sum_{j \in N^\mathrm{out}_i} \rho_{jit}, \, c_{it} = \sum_{j \in N^\mathrm{in}_i} \rho_{ijt}$.
We name $R_{it}$ as profit.

In this case, $\unit$ maximizes the cumulative discounted profit $G_{it}$ represented as
\begin{flalign}
	G_{it}	= \sum_{k=0}^T \gamma^k R_{i, t+k} 
			= \sum_{k=0}^T \gamma^k (r_{i,t+k} - c_{i,t+k})
			&= r_t - c_t + \gamma G_{i,t+1}.
\end{flalign}
$G_{it}$ is unobserved unless the time is reached at the end of the episodes.
Because prediction based on the current value is needed to select the optimal actions, 
we approximate $G_{it}$ with value function $V_i^{\pi_i} (s_{it}) = \Expect{\pi_i}{ G_{it} \mid s_{it}}$ where $s_{it} \in \Observations$.
In this case, the following equation holds.
\begin{flalign} 
		V_i^{\pi_i} (s_{it}) = r_{it} - c_{it} + \gamma V_i^{\pi_i} (s_{i, t+1}),	\label{eq:V}
\end{flalign}
Therefore, we need only consider maximization of revenue, the value function, and cost minimization.
$R_{it} > 0$, i.e., $r_{it} > c_{it}$ indicates that the unit gives the additional value to the obtained data.
The unit acts NOOP because $V_i^{\pi_i} (s_{it}) \leq 0 < \delta$ if $R_{it} \leq 0$ for all $t$.

%TODO: verify equation of V
