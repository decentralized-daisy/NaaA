\section{Conclusion and Future Works}
本論文では、ニューラルネットワーク上の各ユニットをエージェントとして扱うフレームワーク、NaaA について述べた。
最初に、単純に NaaA を最適化しただけではジレンマ問題が存在することについて述べ、
オークション理論を用いた最適化手法について述べた。
その結果、それぞれのエージェントが付加価値を評価する行動が、ナッシュ均衡として得られ、全体としてパレート最適になることを示した。
入札価格の決定アルゴリズムの一つとして、$Q$-learning による手法を提案し、
Adaptive Dropconnect になることを示した。
評価実験では、Atari と VizDoom を用いた実験を行い、実験結果が既存手法よりもよくなることを示した。

% 今後の方向性は思いついたら書き足していく
今後の方向性として、高速化や、価値評価を on-policy な手法で置き換えるといった方向性の他、神経科学的な説明の強化、遺伝的アルゴリズムと組み合わせたハイパーパラメータチューニングへの応用を行っていく予定である。
